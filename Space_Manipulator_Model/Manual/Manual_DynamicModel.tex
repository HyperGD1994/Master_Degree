\documentclass[UTF8]{ctexart}
%\usepackage{ctex}
\usepackage{graphicx}
\usepackage{amsmath}
\usepackage{hyperref}
\usepackage{geometry}
\usepackage{float}

\usepackage[super,square]{natbib}
\geometry{left=2cm,right=2cm,top=2.5cm,bottom=2.5cm}

\CTEXsetup[name={},format={\raggedright\bfseries}]{part}
\CTEXsetup[name={},format={\raggedright\bfseries}]{section}
\CTEXsetup[name={},format={\raggedright}]{subsection}
\CTEXsetup[name={},format={\raggedright}]{subsubsection}
\CTEXsetup[name={},format={\raggedright\bfseries}]{paragraph}
\pagestyle{plain}

\usepackage{CJKulem}
%\usepackage{enumerate}
\usepackage{amssymb}

\renewcommand{\thesection}{\number\numexpr\value{section}-1.\relax}
\renewcommand{\thesubsection}{\number\numexpr\value{subsection}.\relax}


\usepackage{enumitem}


\begin{document}
\title{《空间机器人动力学模型》说明}
\author{弓铎}
\maketitle

\noindent 空间机械臂动力学模型,根据Matlab语言SpaceDyn编写
\begin{itemize}
	\item Python3.6
	\item 只完成前向动力学
	\item 某些变量(构型变量)的存储下标并未从零开始,待标准化
	\item 关节类型只考虑转动关节(rotational),未处理平动关节
	\item 使用多维数组时,多次强制进行形状转换(reshape),可能导致不易察觉的错误,待改进
\end{itemize}

\section{变量说明(self)}

\begin{enumerate}[itemindent=0.3em]  
	\item 构型相关:(connect相关变量存储下标存在问题,并未从0开始)
	\begin{itemize}
		\item link\_num:连杆数目n,编号从0到n-1,其中,0为基座
		\item connect\_lower:低连接,$ 1 \times n-1$数组,connect\_lower[i]表示编号i+1连杆的低连接编号
		\item connect\_upper:高连接,$ n-1 \times n-1 $数组
%		\begin{equation*}
%		connect\_upper[i,j]=\left\{\begin{aligned}
%		-1&i=j\\
%		1&i=conncet\_lower[j]\\
%		0&otherwise
%		\end{aligned}\right.
%		\end{equation*}
		\item connect\_end,connect\_base:是否与末端、基座相连(1是,0否)
	
	\end{itemize}
	\item inertia:惯量矩阵,$ n \times 3 \times 3$数组,第一下标表示连杆编号
	\item mass:质量,分别为连杆质量
	\item q\_from\_Bi\_to\_i:连杆姿态,为坐标系$\sum_{Bi}$ 到坐标系 $\sum_{i}$的欧拉角变换
	\item link\_vector:连杆矢量,$ n+1 \times n+1 \times 3$数组,意义如图
\end{enumerate}

\section{方向余弦和坐标变换}
\begin{enumerate}[itemindent=0.3em]
	\item 坐标系定义:原点固连在关节上,z轴为关节旋转轴(关节角q相关),其余定义可由self.q\_from\_Bi\_to\_i和self.link\_vector确定
	\item 方向余弦,采用如下定义:
	\begin{align*}
		\left\{\sum\nolimits_{i}\right\}&={}^{i}\!C_{i-1}\left\{\sum\nolimits_{i-1}\right\}\\
		&=\left[C_{3}(q_{i})C_{3}(\gamma_{i})C_{2}(\beta_{i})C_{1}(\alpha_{i})\right]^{T}\left\{\sum\nolimits_{i-1}\right\}
	\end{align*}
	\item 坐标变换:$ {}^{I}\!A_{i}=C_{i}^{T}$
\end{enumerate}

\section{运动学}
\begin{enumerate}[itemindent=0.3em]
	\item 计算坐标变换$ A = clac\_coordinate\_transform(A_{0},q) $ 
	\begin{align*}
	A_{i} &= A_{B_{i}}\cdot{}^{i}C_{B_{i}}^{T}\\
	& = A_{B_{i}}\cdot (C_{z}(\gamma+q_{i})C_{y}(\beta)C_{x}(\alpha))^{T}
	\end{align*}
	\item 计算连杆位置$R=calc\_position(R_{0},A,q)$ 
	\begin{align*}
		R_{i} &= R_{B_{i}} + A_{B_{i}}\cdot l_{B_{i},i} - A_{i}\cdot l_{i,i}
	\end{align*}
	\item 计算连杆速度 $ v, \omega = calc\_velocity(A,v_{0},\omega_{0},q,\dot{q})$
	\begin{align*}
		\omega_{i} &= \omega_{B_{i}} + (A_{i}\cdot e_{z} )\dot{q}_{i}\\
		v_{i} &= v_{B_{i}} + \omega_{B_{i}} \times (A_{B_{i}}\cdot l_{B_{i},i}) - \omega_{i}\times (A_{i}\cdot l_{i,i})
	\end{align*}
	\item 计算连杆加速度
\end{enumerate} 

\section{前向动力学}
系统动力学方程写作:
\begin{equation*}
	\left[\begin{matrix}
	\boldsymbol{H}_{b} && \boldsymbol{H}_{bm}\\
	\boldsymbol{H}_{bm}^{T} && \boldsymbol{H}_{m}
	\end{matrix}\right]\left[\begin{matrix}
	\ddot{x}_{b}\\
	\ddot{\boldsymbol{\Theta}}
	\end{matrix}\right]+\left[\begin{matrix}
	c_{b}\\
	c_{m}
	\end{matrix}\right]=\left[\begin{matrix}
	\boldsymbol{F}_{b}\\
	\boldsymbol{\tau}_{m}
	\end{matrix}\right]
\end{equation*}
\begin{enumerate}[itemindent=0.3em]
	\item 计算平动雅克比矩阵 $ J_t = calc\_translational\_jacobian(R,A)$ 
	\begin{align*}
		J_{ti} &= \left[k_1 \times(r_i-p_1),\cdots,k_i \times(r_i-p_i),0,\cdots,0\right]\in \mathbb{R}^{3\times n}\\
		& = (A_j\cdot e_z) \times (R_i - R_j - (A_j \cdot l_{j,j}))\\
	\end{align*}
	(i,j详细计算见程序)
	\item 计算转动雅克比矩阵 $ J_r = calc\_rotational\_jacobian(R,A)$ 
	\begin{align*}
	J_{r i} &= \left[k_1 ,\cdots,k_i ,0,\cdots,0\right]\in \mathbb{R}^{3\times n}\\
	& = (A_j\cdot e_z) \\
	\end{align*}
	(i,j详细计算见程序)
	\item 计算惯量矩阵 $ H = clalc\_inertia\_matrices(R,A)$
	\begin{eqnarray*}
	 J_t &= &calc\_translational\_jacobian(R,A) \\
	 J_r &= &calc\_rotational\_jacobian(R,A)\\
	 \omega E&=& M E_{3\times3}\\
	 \omega r_{0g}&=& (R_g -R_0)M\\
	 J_{tg} &=& \sum m_i J_{ti}\\
	 H_\omega &=& \sum A_i\cdot inertia_i\cdot A_i^T+m_i\tilde{r}_{0i}^T\cdot\tilde{r}_{0i}	 \\
	 H_{\omega q} &=& \sum (A_i\cdot inertia_i \cdot A_i^T)\cdot J_{ri} + m_i\tilde{r}_{0i}\cdot J_{ti}\\
	 H_q &=& J_{ri}^T\cdot A_i\cdot inertia_i \cdot A_i^T\cdot J_{ri} +m_i J_{ti}^T J_{ti}
	\end{eqnarray*}
	
	\begin{equation*}
	\boldsymbol{H}=\left[\begin{matrix}
	\omega E & \tilde{\omega r}_{0g}^T& J_{tg}\\
	\tilde{\omega r}_{0g} & H_\omega & H_{\omega q}\\
	J_{tg}^T &  H_{\omega q}^T& H_q
	\end{matrix}\right]
	\end{equation*}
\end{enumerate}
\begin{table}[h]
	\centering
	\begin{tabular}{cc}
		a&b
	\end{tabular}
\end{table}



\end{document}